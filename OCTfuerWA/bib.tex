\begin{thebibliography}{1}

    
    %SCOOT
    \bibitem{scoot}
    Hunt, P. B. , Robertson, D. I., Bretherton, R. D., Winton, R. I. (1981),SCOOT – A Traffic Responsive
    Method of Coordinating Signals, \emph{Transport and Road Research Laboratory}.
    
    %SCATS
    \bibitem{scats}
    Lowrie, P. R. (1990), SCATS: A Traffic Responsive Method for Controlling Urban Traffic, \emph{Roads and
    Traffic Authority of New South Wales}.

    %Selbst-Steuerung von Lämmer
    \bibitem{laemmer07}
    L\"ammer, S. (2007), Reglerentwurf zur dezentralen Online-Steuerung von Lichtsignalanlagen in Straßennetzwerken, \emph{faculty of traffic sciences "Friedrich List" of Technical University Dresden}.
    
    \bibitem{laemmer08}
    L\"ammer, S., Helbing, D (2008) Self-control of traffic lights and vehicle flows in urban road networks, \emph{ Journal of Statistical Mechanics: Theory and Experiment 2008, April 2008}.
    
    %Organic traffic light control for urban road networks
    \bibitem{organic1}
    Prothmann, H., Branke, J., Schmeck, H., Tomforde, S., Rochner, F., H\"ahner, J., Müller-Schloer, C. (2009) Organic traffic light control for urban road networks, \emph{Karlsruhe Institute of Technology, University Karlsruhe and Institute of Systems Engeneering, Leibniz University Hannover}.
    
    %Quelle für bild "control-loop"
    \bibitem{laemmer09}
    L\"ammer, S., Krimmling, J., Hoppe, A. (2009), Selbst-Steuerung von Lichtsignalanlagen - Regelungstechnischer Ansatz und Simulation, \emph{Stra{\ss}enverkehrstechnik 11, S. 714-721, Kirschbaum Verlag, Bonn}.
    
    \bibitem{simulation}
     Lämmer, S., Krimmling, J., Hoppe, A. (2009) Selbst-Steuerung von Lichtsignalanlagen - Regelungstechnischer Ansatz und Simulation, \emph{Straßenverkehrstechnik 11, S. 714-721.}
    
    \bibitem{laemmerstabil}
    Lämmer, S. (2009) Stabilit\"atsprobleme vollverkehrsabhängiger Lichtsignalsteuerungen, Diskussionsbeitrag, \emph{Technische Universität Dresden}.
    
     %Possibilities and Limitations of Decentralised Traffic Control Systems
    \bibitem{organicHPSS}
    Tomforde, S., Prothman, H., Branke, J., H\"ahner, J., M\"uller-Schloer, C., Schmeck, H. (2010) Possibilities and Limitations of Decentralised Traffic Control Systems, IEEE World Congress on Computational Intelligence
    
     \bibitem{stabilizing}
    Lämmer, S., Helbing, D (2010) Self-Stabilizing Decentralized Signal Control of Realistic, Saturated Network Traffic, \emph{Santa Fe Working Paper Nr. 10-09-019.}
    
     %Self-Organised Routing for Road Networks
    \bibitem{organicRouting}
    Prothman, H., Tomforde, S., Lyda, J., Branke, J., H\"ahner, J., M\"uller-Schloer, C., Schmeck, H. (2012) Self-Organised Routing for Road Networks, \emph{Kralsruhe Institue of Technbology (Kit), Institute AIFB, Leibniz Univerit\"at Hannover, Institute of Systems Engineering, University of Warwick,Warwick Business School}
    
    %QExperiment Selbststeuerung
    \bibitem{laemmer15}
    L\"ammer, S. (2015) Die Selbst-Steuerung im Praxistest, \emph{faculty of traffic sciences "Friedrich List" of Technical University Dresden}.

    %An organic Architecture for Traffic Light Controllers
    \bibitem{organicLCS}
    Rochner, F., Prothmann, H., Branke, J., M\"uller-Schloer, C., Schmeck, H. An Organic Architecture for Traffic Light Controllers, \emph{Institute of Systems Engineering, Universit\"at Hannover, Institute of Applied Informatics and Formal Description Methods, Universit\"at Karlsruhe}
    


\end{thebibliography}
