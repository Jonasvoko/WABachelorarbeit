\section{conclusion}
This paper showcases different approaches to organic traffic control. The first approach uses three layers to control the traffic, change signal plans according to a Learning Classifier System if the current plan is performing below a certain threshold and create new plans if none of the existing plans performs well for the current traffic demand. Afterwards a further development based on the first approach was presented which focus on combining the locally optimised traffic nodes into progressive signal systems. 
While this further development decreased overall stops for vehicles in urban road networks it also added another layer of complexity to the system making it more difficult to implement and maintain. Therefore a compromise should be found for the implementation in a real urban center between the increase in traffic flow allowed by the OTC and the costs involved in implementing these systems.


The second approach does not learn new behaviors like the first. Instead SC relies on a simple optimisation that models incoming traffic streams as pressure and yields to the strongest pressure. From the viewpoint of a traffic operator the SC is a black box he has to trust it, because he can not influence its behavior after the weighting parameters are set. On the contrary that means less maintenance costs. In the user's view the SC can seem chaotic, if he does not know its principles. The traffic participants would need to be cleared up about the SC to accept it. Particularly the possibilities to adjust the waiting weights for different types of traffic and special situations makes hope for further development of SC. But the the same possibilities to adjust the weights are the source for remaining problems with SC. In the Dresden experiment wrong parameters were the reason for a bad performance of some traffic streams. A big problem for SC in the future might be, that it can not support new concepts like routing algorithms and intersection assistants, because it misses communication between notes and reliable prediction for the duration of the current green phase.


Both the OTC and the SC have been successfully tested in realistic environments. But due to different locations and therefore completely different traffic conditions the experimental results can not be compared. Furthermore, the experiments mostly measured increased or decreased waiting times to an already existing controlling architecture like a fixed time controller in the Hamburg experiment or a VS-PLUS in Dresden. Meaning that the results must be seen relative to the previous systems, which were unique for every experiment. This further increases the difficulty in comparing OTC and SC. A first step to compare the different approaches would be to simulate the architectures by a software in the same traffic network.

