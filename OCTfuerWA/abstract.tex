\begin{abstract}

In recent years new methods for urban traffic control have been developed. Some of them are using the principles of organic computing. A controlling system is called organic if multiple local controls create an adaptive and smart behavior, when in interaction. This paper introduces different organic approaches to traffic control. The first two sections are discussing the motivation to improve traffic control architectures and outlining traditional methods of urban traffic control. In the third section a first approach to organic traffic control is introduced. This approach uses different layers and needs some time to manage an intersection optimally. Further developments of the same approach are the topic of the third section too.
After that another organic approach will be discussed in the fourth section. This time the approach is to model incoming traffic streams as pressure and favor the stream with the most pressure. The final and fifth section serves as conclusion.


\end{abstract}