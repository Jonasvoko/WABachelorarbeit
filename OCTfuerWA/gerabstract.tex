\begin{abstract}

In den letzten Jahren wurden neue Methoden f\"ur die Verkehrsregelung in urbanen Gegenden entwickelt. Einige von diesen Methoden basieren auf den Prinzipien von Organic Computing. Ein System wird organisch genannt, wenn mehrere lokale Steuerungen ein intelligentes und anpassungsfähiges Verhalten erzeugen, w\"ahrend sie miteinander kommunizieren. Dieses Paper stellt verschiedene Herangehensweisen an organische Verkehrsregelung vor. Die ersten beiden Kapitel befassen sich mit den Beweggr\"unden f\"ur die Verbesserung der Verkehrsregelung und stellen traditionelle Methoden vor. Im dritten Kapitel wird eine erste Herangehensweise an organische Verkehrregelung vorgestellt. Dieser Entwurf verwendet verschiedene Ebenen und braucht einige Zeit, bis der Verkehrfluss an einer Kreuzung optimiert ist. Des Weiteren werden im dritten Kapitel Entwicklungen basierend auf dem vorgestellten Entwurf beleuchtet.
Im vierten Kapitel wird ein anderer Ansatz pr\"asentiert. Diesmal wird dabei der Verkehr als Belastung der jeweiligen Strasse modeliert und zu Gunsten der gr\"ossten Belastung entschieden. Im letzen Kapitel gibt es ein Fazit.


\end{abstract}