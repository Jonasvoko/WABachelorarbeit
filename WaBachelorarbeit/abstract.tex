\begin{abstract}


Only in the last few years the research of simple leafs has gotten popularity. But in times of annual changes between summer and winter it is of vital importance to gain an understanding of the objects in close vicinity to leafs. First research results have lead to extensive understanding about the influence the emotional state of a leaf has on its surroundings. This paper is discussing an approach to safely recognize the emotional state of a leaf. This will help to understand how leafs manipulate their surroundings. To recognize the emotional state we will be studying the coarse and fine structure of a leaf and offer some categorizations for said structure.


Die Erforschung von gemeinen Blättern ist erst in den letzten Jahren zu großer Bekanntheit und großer populariät gekommen. In Zeiten des jährlichen Wechsels zwischen Sommer und Winter ist es jedoch ein wesentliches Thema um ein Verständnis über die jeweiligen Objekte in der nähreren Umgebung von Blättern zu erhalten. Erste Forschungsergebnisse haben zu weitreichenden Erkenntnissen in Bezug auf die Auswirkung des Gemütszustandes eines Blattes auf seine Umwelt geführt. In dieser Arbeit soll nun ein Anstaz erklärt werden, wie man den Gemütszustand eines Blattes sicher bestimmen kann um so tiefgreifende Erkenntis über die von Blättern ausgeführte Manipulation ihrer Umwelt zu erfahren. Hierfür wird insbesondere die grob und Feinstruktur verschiedener Blätter betrachtet und einige Kategorisierungen in Bezug auf die Struktur vorgenommen.

\end{abstract}