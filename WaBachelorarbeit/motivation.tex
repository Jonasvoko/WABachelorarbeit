\section{Motivation}
More and more vehicles are driving on the streets in our growing cities. This is why it gets increasingly more important to implement traffic control architectures that aim to reduce waiting time for every traffic participant. To reduce waiting time, means to decrease environmental pollution, reduce fuel consumption and to save time. To achieve all that, the control of traffic lights needs to be optimised to the local conditions. Some of those conditions are set like the structure of the road network itself and speed limits. Others can change dramatically. Examples would be the number of vehicles or the direction the vehicles are coming from. Those conditions can change by time of day, season or to special events such as rock concerts or catastrophes. Therefore a traffic control architecture can only be optimal if it is adapting to the situation at hand.

Furthermore, new technologies like intersection assistants are developed by automobile manufacturers. These new technologies need to communicate with the traffic control system. So traffic control systems should be able to give information to traffic participants by design. Otherwise a technology like the intersection assistant cannot reach its full potential.